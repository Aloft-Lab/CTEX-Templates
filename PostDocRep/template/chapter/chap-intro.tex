
\chapter{引言}
\label{chap:introduction}

\texttt{PostDocRep}~宏包是根据博士后研究报告撰写要求编写的。
宏包的目的是简化博士后研究报告的撰写,使得作者可以将精力集中到
报告的内容上而不是浪费在版面设置上。同时宏包在符合研究报告
撰写要求的基础上尽可能地进行美化,其中还参考了出版界的一些排版规范。

\section{系统要求}

\texttt{PostDocRep}~宏包可以在目前大多数的~\TeX{}~系统中使用,例如~C\TeX{}、
~MiK\TeX{}、~te\TeX{}、~fp\TeX{}。

\texttt{PostDocRep}~宏包通过~\texttt{ctex}~宏包来获得中文支持。~\texttt{ctex}~
宏包提供了一个统一的中文~\LaTeX{}~文档框架,底层支持~CCT~和~CJK~两种中文~\LaTeX{}~系统。
最新的~\texttt{ctex}~宏包可以从~\url{http://www.ctex.org}~网站下载。

此外,~\texttt{PostDocRep}~宏包还使用了宏包~amsmath、~amsthm、~amsfonts、
~amssymb、~bm~和~hyperref。目前大多数的~\TeX{}~系统中都包含有这些宏包。

最新的~C\TeX{}~套装(2.4.3~以上版本)中包含了以上列出的各种宏包,用户无需额外的设置即可使用。

\section{下载与安装}

\texttt{PostDocRep}~宏包的最新版本可以从~\url{http://www.ctex.org}~网站下载。
C\TeX{}~套装每次更新时都将会包含最新版本的~\texttt{PostDocRep}~宏包。

\texttt{PostDocRep}~宏包包含两个文件:~\texttt{PostDocRep.cls}~和~\texttt{PostDocRep.cfg}。
简单方便的安装方法是将宏包文件和论文~\texttt{.tex}~文件放置在同一目录下。
或者将宏包文件放置到~\TeX{}~系统的~localtexmf/tex/latex/PostDocRep~目录下,
然后刷新~\TeX{}~系统的文件名数据库。

同时,宏包还提供了一个使用模板,也就是这份说明文档的源文件。用户可以通过修改
这个模板来编写自己的学位论文。

用户也可以下载宏包源文件~\texttt{PostDocRep.dtx}~和~\texttt{PostDocRep.ins}~,
然后对~\texttt{PostDocRep.ins}~文件运行~\texttt{latex}~编译命令来得到宏包文件。

关于安装过程的问题可以参考~C\TeX{}-FAQ~以及其他~\LaTeX{}~教材。

\section{宏包定制}

\texttt{PostDocRep}~宏包的设置都保存在~\texttt{PostDocRep.cfg}~文件中。
用户可以在~\texttt{.tex}~中通过宏包提供的命令修改设置。对于常用的设置修改,
如单位名称、专业名称等,可以直接在~\texttt{PostDocRep.cfg}~文件中进行。
各单位可以修改后提供本单位统一的~\texttt{PostDocRep.cfg}~文件供本单位用户使用。

\section{问题反馈}

用户在使用中遇到问题或者需要增加某种功能,都可以和作者联系:

\begin{center}
吴凌云 (aloft) \quad \href{mailto:aloft@ctex.org}{aloft@ctex.org}
\end{center}

欢迎大家反馈自己的使用情况,使我们可以不断改进宏包。
